\chapter{Introdução}

Perdas auditivas podem tonar-se um fator limitante na sociabilização de um indivíduo, existindo como contrapartidas métodos baseados em aparelhos de amplificação sonora individual (AASI), para perdas leves e implantes cocleares (IC) para perdas de grau severo e profundo.

A partir dos critérios definidos por portarias ministeriais atualizados em 2014 elevou-se o número de pacientes que, por meio de avaliação utilizando tais critérios, tem direito a implante coclear. Contudo testes de reconhecimento de fala utilizados atualmente não levam em conta a distribuição espacial de sinal e de ruído, podendo levar a indicações não necessárias ou excluindo indicações que seriam realmente necessárias.

Este estudo visa padronizar e aprimorar o protocolo de avaliação utilizado atualmente na rede pública de saúde levando em conta a distribuição espacial, e o conteúdo espectro temporal dos testes. 

As faixas etárias para os testes são divididas entre, x e x e x e x, onde a maior população encontra-se na faixa etária tal. Assim o foco do projeot é para essa faixa

\chapter{Organização Inicial}
A seguir algumas etapas e dúvidas para a discussão:

\section{Projeto/Construção da cabine}

A proposta da cabine é ser multifuncional, para testes variados, que tanto atendam requisitos normativos para cabines audiométricas bem como possa servir para demais testes dentro do LVA\footnote{Sigla para Laboratório de Vibrações e Acústica}.


\subsection{Projeto da cabine}
Projetar a cabine com especificações desejadas e para além dos testes de percepção de fala é o primeiro plano.

Algumas especificações iniciais como dimensões e necessidades especiais, como isolamento eletromagnético e níveis de pressão sonora no interior da cabine gerado pelo sistema de ventilação foram levantados a partir das normas que regem as cabines audiométricas e da necessidade de acomodar no espaço do LVA.

\begin{itemize}[noitemsep]
    \item Dimensões iniciais: Altura 2,18 m x Largura 1,97 m x Comprimento 2,35 m
    
    \item Peso estimado: 1.500 a 2.000 Kg 
    \item Isolamento eletromagnético de 86 MHz\footnote{Faixa de Rádio VHF (86-108 MHz)} a  2484 MHz\footnote{Faixa de WLAN (\textit{Wireless Local Area Network}) e \textit{Bluetooth data transfer} (2400 - 2484 MHz)}
    \item Normas: 
    \begin{itemize}
        \item ISO 8253-1:2010 \textit{Acoustics -- Audiometric test methods} 
        \item  Resolução 364 do C.F.Fª.
        \item ANI S3.1-1999 \textit{Maximum Permissible Ambient Noise Levels for Audiometric Test Rooms}
    \end{itemize}
    
    \item Sala LVA para alocação da cabine.
    
\end{itemize} 

\textbf{Adaptação e utilização de cabine}

Como alternativa uma cabine de um centro aplicador pode ser adaptada para o teste com multi canais.
\section{Levantamento de protocolos atuais}
Realizar levantamento do protocolo utilizado pelos centros aplicadores:

Através de questionários \textit{on-line} quantificar e qualificar os testes e os métodos de aplicação de testes de reconhecimento de fala na presença de ruído

Perguntas e Hipóteses pesquisa
Viva voz, gravado, tipo de ruído, 

\begin{itemize}
    \item Listas da ABA
    \item Contatos via Luciana
    \item Contatos via Izabel
    \item Protocolos utilizados, (Quais listas) Como fazer esse levantamento? (questionários on-line para clínicas)
    \begin{itemize}
    \item Tempo de duração médio do teste
    \item Protocolos atuais levam em conta distribuição espacial entre som e ruído? (pode ser uma questão aplicada às clínicas?)
    \item Quais tipos de ruído são utilizados (outra questão)
    \item Quais ruídos serão considerados/propostos? 
    \end{itemize}
\end{itemize}

\section{Adaptação do protocolo}

A partir dos testes mais recorrentes, avaliar com parâmetros objetivos de sinal os mais adequados? 

%Há um artigo sobre equivalência das listas da Maristela (equivalência entre listas) poderia ser utilizado em paralelo para avaliar a equivalência entre listas de testes diferentes talvez?

Sugestão de hipótese (Voz masculina interfere no resultado com menor índice de acerto): Teste comparativo entre locutor masculino e feminino.
\begin{itemize}
    \item Comparação entre listas
    \item Gravações calibradas * Sinais de fala 
    \item Espacialização  
\begin{itemize}
    \item Estimativa de LA$_\text{eq}$ na apresentação multicanal
    \item 
\end{itemize}
    \item Ruídos  
    \begin{itemize}
    \item Definição tipos
    \item Gravação calibrada
\end{itemize}

    \item Quantidade de apresentações e tamanho total do teste
    \item Construção do algoritmo para apresentação multicanal
    \item Adaptação para apresentação 2 ch (virtualizar demais - integração com audiômetro) 
    \item Definição de teste beta
\end{itemize}
\pagebreak
a
%\section{EIA}

% Dia 26

% \begin{itemize}
%     \item 10:30-11:00 Protocolo clínico para indicação e adaptação do Sistema FM: como você faz? Regina Tangerino de Souza Jacob (SP)
    
%     \item 11:30-12:00 Protocolo de avaliação  audiológica pediátrica - Doris Ruth Lewis (SP)
    
%     \item 14:00-14:30 Novos estímulos acústicos na Triagem Auditiva Neonatal: aumentando a eficácia - Mabel Gonçalves Almeida (ES) (Sala E)
    
%     \item 14:30-15:00 Experiência ambulatorial de um Programa de Triagem Auditiva - Maria Cecília Marconi Pinheiro Lima (SP) (Sala E)
    
%     \item 15:00-15:30 Triagem auditiva em crianças e escolares: indicadores e procedimentos -  Ana Claúdia Vieira Cardoso (SP) (Sala E)
    
%     \item 15:30-16:00 Avaliação do comportamento auditivo procedimento na avaliação audiológica infantil -  Marisa Frasson de Azevedo (SP) (Sala E)
    
%     \begin{itemize}
%         \item 14:00-16:00 Fórum 2 Aparelho de Amplificação Sonora Individual
% Temas:
% \textbf{Critérios para
% indicação de aparelhos
% de amplificação sonora individual para perdas auditivas unilaterais/assimétricas
% }\\ \textbf{Aprovação das Diretrizes para Seleção, Indicação e Adaptação de Aparelhos
% de Amplificação Sonora Indivual para adultos, idosos e crianças}
%     \end{itemize}
% \end{itemize}

% Dia 27

% \begin{itemize}
%     \item 08:00-09:30 (SALA B) - Mesa Redonda Internacional Tema:
% Implante coclear bilateral Andreas Beynon (Holanda) - Maria Valéria Schimidt Gof  (SP)

% \item 09:00-09:30 (SALA G) - Índice de Reconhecimento Máximo de fala para monossílabos: existe uma intensidade ideal para pesquisá-lo?

% \item 10:30-11:00 (SALA G) - Protocolo para avaliação comportamental do Processamento Auditivo

% \item 11:00-11:30 (SALA G) Atualidades em percepção de fala em ruído

% \item 11:30-12:00 (SALA F) Equipamentos de custo acessível para triagem auditiva automática em escolares

% \item 12:00-12:30 (SALA G) valiação do processamento auditivo em idosos: que cuidados tomar?

% \item 12:00-12:30 (SALA I) Implante coclear: quais as perspectivas para o futuro

% \begin{itemize}
%     \item 10:30-12:30 SALA H - ÁREA DIAGNÓSTICO AUDIOLÓGICO
% \end{itemize}
% \end{itemize}