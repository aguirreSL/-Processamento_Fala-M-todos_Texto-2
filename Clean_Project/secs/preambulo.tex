%%%%%%%%%%%%%%%%%%%%%%%%%%%%%%%%%%%%%%%%%%%%%%%%%%%%%%%%%%%%%%%%%%%%%%%%%%%%%%%%%%%%%%%%%%%%%%%%%%%%%%%%%%%%%%%%%%%%
%%% Template para relatório, trabalho, manual, artigo, tcc, dissertação ou tese
%%% Release 13/11/2014
%%%	Por Prof. William D. Fonseca
%%% will.fonseca@eac.ufsm.br
%%%%%%%%%%%%%%%%%%%%%%%%%%%%%%%%%%%%%%%%%%%%%%%%%%%%%%%%%%%%%%%%%%%%%%%%%%%%%%%%%%%%%%%%%%%%%%%%%%%%%%%%%%%%%%%%%%%%

%%%%%%%%%%%%%%%%%%%%%%%%%%%%%%%%%%%%%%%%%%%%%%%%%%%%%%%%%%%%%%%%%%%%%%%%%%%%%%%%%%%%%%%%%%%%%%%%%%%%%%%%%%%%%%%%%%%%
%%%%%%%%%%%%%%%%%%%%%%%%%%%%%%%%%%%%%%%%%%%%%%%%%%%%%%%%%%%%%%%%%%%%%%%%%%%%%%%%%%%%%%%%%%%%%%%%%%%%%%%%%%%%%%%%%%%%
%%%======================= Classe
%%% Não mexer neste parte %%%%%%%
%\documentclass[a4paper,12pt]{article}
%\documentclass[a4paper,12pt]{scrreprt} 
%\documentclass[a4paper,12pt,twoside]{report}
\documentclass[a4paper,12pt,twoside,openright]{scrreprt}
% Aqui classe especifica o tipo de documento a ser criado.
% Gerar o documento como um report com a fonte base de onze pontos,usando papel A4.
% {report} Para relatórios maiores contendo diversos capítulos, pequenos livros, teses de doutorado,...
% [twoside] paginas impares e pares

%%%%%% Watermark %%%%%%%%%%%%%%%%%%%%%%%%%%%%%%%%%%%%%%%%%
%\usepackage{draftwatermark}
%\SetWatermarkText{Confidencial}
%\SetWatermarkText{\includegraphics[width=1\linewidth]{water.pdf}}
%\SetWatermarkColor[gray]{0.92}
%\SetWatermarkScale{0.7}

%%%%%%%%%%%%%%%%%%%%%%%%%%%%%%%%%%%%%%%%%%%%%%%%%%%%%%%%%%
% Check if the compilation is a PDF
\usepackage{ifpdf}   
\usepackage{listingsutf8} 
\usepackage[colorinlistoftodos]{todonotes}
\newcommand{\note}[1]{\todo[inline]{#1}}

%% To enable copy/paste from pdftex-generated PDF,
\ifpdf
\input{glyphtounicode}
\pdfgentounicode=1\fi
%%%%%%%%%%%%%%%%%%%%%%%%%%%%%%%%%%%%%%%%%%%%%%%%%%%%%%%%%%
\usepackage{scrhack}
% Ative caso use ``scrreprt'', este pacote evita conlfito com o hyperref

% Ajustando magerns para padrao % for easy management of document margins
\usepackage[a4paper,left=2.3cm,right=2.0cm,top=2.8cm,bottom=2.2cm]{geometry}

\usepackage{chngpage} % to easily change the margins of pages.
											% Caso precise folhas com margens especiais
\usepackage{pdflscape} % Use landscape pages	
%%%%%%%%%%%%%%%%%%%%%%%%%%%%%%%%%%%%%%%%%%%%%%%%%%%%%%%%%%
%%%======================= Entrada e Fontes
%\usepackage{ucs} % Provides advanced support for using UTF-8 as the input encoding of LaTeX files.

% Para configurar a codificação do arquivo de entrada permite usar os caracteres acentuadas.
\usepackage[utf8]{inputenc} % please do not change! 

%\usepackage[ansinew]{inputenc} 
% Ansinew and cp1252 (they are synonyms) for Windows 3.1 ANSI (an MS extension of ISO Latin-1)
% In Emacs cp1252 is an alias of windows-1252

% Para configurar a codificação do arquivo de entrada. 
% Permite o uso em plataformas Windows, Linux, Unix e Mac, porém não é possível escrever acentos, e.g. áéó.
%\usepackage[utf8]{inputenc} 
%\inputencoding{latin1} % To switch to Latin 1

\usepackage[english,brazilian]{babel} % The multilingual package. 
% O pacote babel especifica a linguagem/dialeto utilizada. O último torna-se o padrão
% Ele acerta a hifenizacao e titulo/nome que ele coloca automaticamente.

\usepackage[T1]{fontenc} 
\usepackage{textcomp}
\usepackage{tipa}
\usepackage{fourier-orns}
\usepackage{lmodern}
% \usepackage{listingsutf8} 

\usepackage{setspace} % \sin­glespac­ing, \one­half­s­pac­ing, and \dou­blespac­ing com­mands
\usepackage{lipsum} %% Create dummy text 
%%% Habilita Palatino %%%%%%%%%%%%%%%%%%%%%%%%%%%%%%%%%%%%
%\usepackage[sc]{mathpazo}
%\usepackage{tgpagella}
%%% Habilita similar a Times New Roman %%%%%%%%%%%%%%%%%%%
%\usepackage{mathptmx} % http://ctan.org/pkg/mathptmx
%%%%%%%%%%%%%%%%%%%%%%%%%%%%%%%%%%%%%%%%%%%%%%%%%%%%%%%%%%
%%%======================= Espaco adicional no listofigures
\makeatletter
\renewcommand*\l@figure{\@dottedtocline{1}{1.5em}{2.9em}}
\makeatother
%The length 1.5em is the space from the left margin to every line in the list of figures, whereas 2.3em is the space reserved for the number of every figure. 
%It seems that you need to increase this length. Put in the preamble of your document something like:
%%%%%%%%%%%%%%%%%%%%%%%%%%%%%%%%%%%%%%%%%%%%%%%%%%%%%%%%%%
%%%======================= Matematica
\usepackage{amsmath, amssymb, amsfonts, xparse}
% amsmath carrega diversos pacotes do AMS para incrementar o ambiente matematico.
% Os pacotes amssymb e amsmath sao mais usados
% amssymb Carrega diversos simbolos matematicos adicionais. 
% Este pacote carrega automaticamente o amsfonts 

\usepackage{amsthm}
%The amsthm package provides an enhanced version of LATEX's \newtheorem command
%for defining theorem-like environments.

\usepackage{array}
\usepackage{subeqnarray}
% Como visto anteriormente,cada equação recebe uma diferente referência.Porém,se o
% usuário desejar usar a mesma referência para todas as equações é só utilizar o pacote chamado
% subeqnarray. That the individual lines are numbered like 1a, 1b, 1c, etc.
% Isto é: coloca numeracao 3.7a , 3.7b , etc ....nas equacoes, usar \slabel (ex: \slabel{sub2}a&=&b-5)

\usepackage{chemarrow} %Seta em cima da seta

\usepackage{mathrsfs}
% Support use of the Raph Smith's Formal Script  font in mathematics. Provides a \mathscr command, rather than overwriting the standard \mathcal command, as in calrsfs.
\usepackage{bigints,nicefrac} 

\usepackage{etoolbox} %%% Correct sub-equation
\patchcmd{\subequations}{\def\theequation{\theparentequation\alph{equation}}}%
{\def\theequation{\theparentequation.\arabic{equation}}}{}{}	
% More math fonts
\usepackage{upgreek} \usepackage{dsfont} %\usepackage{latexsym}
%%%%%%%%%%%%%%%%%%%%%%%%%%%%%%%%%%%%%%%%%%%%%%%%%%%%%%%%%%
%%%======================= Imagens 
\usepackage{graphicx,wrapfig}
% Pacote padrão para incluir figuras, usando comando \includegraphics[]{}.
\usepackage{animate} % Para compilar animações

% Figuras.pdf Se opacote graphicx for usado como opcional[pdftex]fica possível inserir
% figuras no formato *.pdf,neste caso o documento não poderá ser compilado como latex e
% sim como pdfatex. Deve-se conferir se seu sistema oferece este recurso.
% Ex: \usepackage[pdftex]{graphicx}
% ver apostila_latex.pdf Figuras.jpg,.png,.pdf 

\usepackage[absolute]{textpos}
\usepackage[format=plain,justification=centerlast,margin=4pt]{caption} %para fazer captionof
\captionsetup[table]{font=normalsize,skip=10pt}
\captionsetup[quadro]{font=normalsize,skip=8pt}
\captionsetup[figure]{font=normalsize,skip=8pt}
\captionsetup{belowskip=2pt,aboveskip=4pt}

%\setlength{\captionmargin}{10pt}

%%%These package defines the environment sidewaystable, it must be loaded after the graphicx package
\usepackage{rotating}   % Package for rotating things
\usepackage{calc} 			% ? precisa para o newenvironment{onde}, junto com o amsmath
% calc, pacote para efetuar cálculo aritmético, útil para calcular comprimento e valores de contadores.
% ex: altura de letra, comprimento de palavra
%\setcounter{x}{7/2}
%\setcounter{y}{3*\real{1.6}}
%\setcounter{z}{3*\real{1.7}}
%will assign the value 3 to the counter x, the value 4 to y, and the value 5 to z.
%This truncation also applies to intermediate results in the sequential computation
%of a composite expression; thus, the following command
%%%%%%%%%%%%%%%%%%%%%%%%%%%%%%%%%%%%%%%%%%%%%%%%%%%%%%%%%%
%%% Tabelas e Quadros
\usepackage{array,tabularx}
% Array: Arrays e tables (tabular) com colunas formatados
% Tabularx: tabulars that widen automatically. 
% Para igualar colunas desejadas do array/table, mantendo o cálculo automatico para largura.
\usepackage{multirow}
%%%% Cria o tipo de objeto quadro
\usepackage{newfloat} 
\DeclareFloatingEnvironment[
    fileext=loq,
    listname={Lista de Quadros},
    name=Quadro,
    placement={H},
    within=chapter,
]{quadro}
\renewcommand{\listofquadros}{\listof{quadro}{\listquadroname}}
%%%%%%%%%%%%%%%%%%%%%%%%%%%%%%%%%%%%%%%%%%%%%%%%%%%%%%%%%%
%%%======================= Texto
\usepackage{color}         											% para letras e caixas coloridas
%\usepackage[usenames,dvipsnames]{xcolor}        % para letras e caixas coloridas possivel usar RGB, CMYK, etc

\usepackage{indentfirst} 												% Indenta a primeira linha do section/chapter

%\usepackage{multicol, wasysym, xspace, fancyhdr, shapepar}
\usepackage{multicol, wasysym, xspace, fancyhdr, shapepar, float}

\usepackage{subfig}
\DeclareSubrefFormat{myparens}{#1~(#2)}
\captionsetup[subfloat]{subrefformat=myparens,labelformat=parens,position=bottom,font={small,rm}}
%\renewcommand{\thesubequation}{\themainequation.\alph{equation}}

% To correct continued float. Use \resetsubfigs after \begin{figure} to reset the subfigure labels.
\makeatletter
\newcommand\resetsubfigs{\setcounter{sub\@captype}{0}}
\makeatother

% - float, Improves the interface for defining floating objects such as figures and tables.
% - subfigure, Provides support for the manipulation and reference of small or ‘sub’ figures and tables within  % a single figure or table environment. 
% - Multicol defines a multicols environment which typesets text in multiple columns (up to a maximum of 10), and %(by default) balances the end of each column at the end of the environment.
% - wasysym, The WASY2 (Waldi Symbol) font by Roland Waldi provides many glyphs like male and female symbols  
% and astronomical symbols, as well as the complete lasy font set and other odds and ends. 
% - xspace, controla o espaço de acordo com o que segue, útil para escrever macro.
% - fancyhdr, controle extensivo sobre cabeçalho e rodapé da página. Cabeçalho e rodapeh de forma sofisticada.
% - shapepar is a macro to typeset paragraphs in a specific shape. Molda o texto em uma certa forma.
% \; \, \: sao tipos de espaçamento. \! eh um espaço negativo. \quad \qquad esçamento positivo maior

%%%%%%%%%%%%%%%%%%%%%%%%%%%%%%%%%%%%%%%%%%%%%%%%%%%%%%%%%%
%%%======================= URL 
\usepackage{url}	% With the url package (\usepackage{url}) you can enter URLs like this: \url{http://www.stack.nl/~jwk/latex/}. 
% Chamar PDF em pagina especifica \href{MasterF.pdf#page.32}{here}
%%%%%%%%%%%%%%%%%%%%%%%%%%%%%%%%%%%%%%%%%%%%%%%%%%%%%%%%%%
\usepackage{enumerate}

%%%======================= Latex
\usepackage{makeidx}
% para criar índice remissivo. Coloque o comando \makeindex no preamble e use \index{} para adicionar um ítem %no indice remissivo. O \printindex serve para colocar o índice remissivo no local desejado. Note que precisará %a seguinte sequência de execussão: latex, makeindex, latex, latex para criar corretamente o índice remissivo %(caso use bibtex também, a sequência será algo como latex, makeindex, bibtex, latex, latex)
% gera indices em ordem alfabetica, 
\makeindex % faça o makeindex

\usepackage{ifthen,xifthen}
% Provides commands of the form `if...then do... otherwise do...'. 
% ex: se a palavra tive um comprimento maior que X, faça isso.
%%%%%%%%%%%%%%%%%%%%%%%%%%%%%%%%%%%%%%%%%%%%%%%%%%%%%%%%%%
%%%======================= Cabeçalho
\def\shortauthor#1{\def\cabei{#1}}
\def\shorttitle#1{\def\cabeii{#1}}
\def\versao{Ver 0.2}

% \def nao da conflito entre arquivos multiplos
% definicoes para rechamar informacoes. Neste caso, dados que irao aparecer no cabeçalho.
\newcommand{\Index}[1]{#1\index{#1}} 
%%%%%%%%%%%%%%%%%%%%%%%%%%%%%%%%%%%%%%%%%%%%%%%%%%%%%%%%%%%%%%%%%%%%%%%%%%%%%%%%%%%%%%%%%%%%%%%%%%%%%%%%%%%%%%%%%%%%
%%% Start PDF metadata
%%%%%%%%%%%%%%%%%%%%%%%%%%%%%%%%%%%%%%%%%%%%%%%%%%%%%%%%%%%%%%%%%%%%%%%%%%%%%%%%%%%%%%%%%%%%%%%%%%%%%%%%%%%%%%%%%%%%
\usepackage{datetime} % to get automaticallly the date
\ifpdf
\pdfinfo
{  /CreationDate (D:\pdfdate) %D:YYYYMMDDHHmmss
   /ModDate (D:\pdfdate)	
} \fi

\shorttitle{ \mbox{} }
\shortauthor{ \mbox{} }
%%%%%%%%%%%%%%%%%%%%%%%%%%%%%%%%%%%%%%%%%%%%%%%%%%%%%%%%%%%%%%%%%%%%%%%%%%%%%%%%%%%%%%%%%%%%%%%%%%%%%%%%%%%%%%%%%%%%
%%%%%%%%%%%%%%%%%%%%%%%%%%%%%%%%%%%%%%%%%%%%%%%%%%%%%%%%%%%%%%%%%%%%%%%%%%%%%%%%%%%%%%%%%%%%%%%%%%%%%%%%%%%%%%%%%%%%
%%%======================= Cite 
\usepackage[numbers,sort&compress]{natbib}
\usepackage{longtable}

%The natbib package sorts the numbers and detects consecutive sequences, so creating “[2–4,6]”.
%%%%%%%%%%%%%%%%%%%%%%%%%%%%%%%%%%%%%%%%%%%%%%%%%%%%%%%%%%
%%%======================= Níveis 
\makeatletter
\setcounter{secnumdepth}{4} %%%% para 4 níveis enumerados
\setcounter{tocdepth}{3}    %%%% para fazer os 4 níveis aparecer no toc
\makeatother
%%%%%%%%%%%%%%%%%%%%%%%%%%%%%%%%%%%%%%%%%%%%%%%%%%%%%%%%%%
%%%======================= PDF functions 
\pdfminorversion=7     % Faz com que não apareça a mensagem de warning sobre a versão do pdf quando compilado.
\pdfimageresolution300
\pdfcompresslevel8	   % Nivel de compressao do PDF

\ifpdf\usepackage{pdfpages}\fi	 %Incluir arquivos de PDF

\usepackage{lscape} 	 %Supplies a landscape environment, and anything inside is basically rotated. No actual page dimensions are changed. Using pdflscape instead of lscape when generating a PDF document will make the page appear right side up when viewed: the single page that is in landscape format will be rotated, while the rest will be left in portrait orientation.


%%% Links
\usepackage[hyperindex,pagebackref=true,pdfusetitle]{hyperref}

%\hypersetup{bookmarks,colorlinks,breaklinks,plainpages=false,pdfpagelabels,linktocpage}
\hypersetup{plainpages=false,colorlinks,breaklinks,linktocpage,hypertexnames=true,naturalnames=false}
\hypersetup{linkcolor=red,citecolor=blue,filecolor=blue,urlcolor=blue} %Colorful
%\hypersetup{linkcolor=black,citecolor=black,filecolor=black,urlcolor=black} %Not Colorful
 
%%%%% Back references language
%\renewcommand*{\backref}[1]{} % No back references
%\usepackage{backrefEN} % English
\usepackage{backrefBR} % Português


% A seguinte linha evita algumas mensagens de warning por parte do hyperref.
\pdfstringdefDisableCommands{\edef\uppercase{}}
%%%%%%%%%%%%%%%%%%%%%%%%%%%%%%%%%%%%%%%%%%%%%%%%%%%%%%%%%%%%%%%%%%%%%%%%%%%%%%%%%%%%%%%%%%%%%%%%%%%%%%%%%%%%%%%%%%%%
%%%%%%%%%%%%%%%%%%%%%%%%%%%%%%%%%%%%%%%%%%%%%%%%%%%%%%%%%%%%%%%%%%%%%%%%%%%%%%%%%%%%%%%%%%%%%%%%%%%%%%%%%%%%%%%%%%%%
%%% EDIT HERE PDF INFO %% Informar titulo, autores, ...
%%%%%%%%%%%%%%%%%%%%%%%%%%%%%%%%%%%%%%%%%%%%%%%%%%%%%%%%%%%%%%%%%%%%%%%%%%%%%%%%%%%%%%%%%%%%%%%%%%%%%%%%%%%%%%%%%%%%
\title{Trabalho prático de processamento Implementação e análises objetiva e subjetiva de  técnicas  de  processamento  para  redução de ruído.}
\author{Sergio Aguirre, Marcio Costa}

\hypersetup{
  pdfinfo={
	 Subject={Análise espectral, Processamento de voz},
   Keywords={Análise espectral, Processamento de voz}
  }
}
%%%%%%%%%%%%%%%%%%%%%%%%%%%%%%%%%%%%%%%%%%%%%%%%%%%%%%%%%%%%%%%%%%%%%%%%%%%%%%%%%%%%%%%%%%%%%%%%%%%%%%%%%%%%%%%%%%%%
%%%%%%%%%%%%%%%%%%%%%%%%%%%%%%%%%%%%%%%%%%%%%%%%%%%%%%%%%%%%%%%%%%%%%%%%%%%%%%%%%%%%%%%%%%%%%%%%%%%%%%%%%%%%%%%%%%%%
%%%%%%%%%%%%%%%%%%%%%%%%%%%%%%%%%%%%%%%%%%%%%%%%%%%%%%%%%%%%%%%%%%%%%%%%%%%%%%%%%%
\usepackage[colorinlistoftodos]{todonotes}
\renewcommand{\note}[1]{\todo[inline]{#1}}

%%% Nomenclature 
\usepackage{nomecl_br_will}	  % Brazilian
%\usepackage{nomecl_en_will} % English
\makenomenclature  % write/run it in the configuration of your compiler every time you want to print a new nomenclature.

\usepackage{Will}  		 % Brazilian
%%%%%%%%%%%%%%%%%%%%%%%%% Insert source codes e.g. Latex, Matlab, Fortran, LabVIEW
\usepackage{Codes2Latex}
\usepackage{enumitem}
%% Pre-load languages
\lstloadlanguages{WMatlab,WFortran,WLatex,WLabview,ttcCode}

%%%%%%%%%%%%%%%%%%%%%%%%%%%%%%%%%%%%%%%%%%%%%%%%%%%%%%%%%%%%%%%%%%%%%%%%%%%%%%%%%%
%%% Personalizações de texto
%%% Insira aqui um ``pacote'' de comandos de texto para facilitar a sua vida ao 
% escrever um documento
\usepackage{Will_tx}  		 
%%%%%%%%%%%%%%%%%%%%%%%%%%%%%%%%%%%%%%%%%%%%%%%%%%%%%%%%%%%%
